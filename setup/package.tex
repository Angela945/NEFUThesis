%!Tex Program = xelatex
% -*-coding: utf-8 -*-


\usepackage{graphicx}

%\usepackage[a4paper,text={162true mm,249true mm},left=24true mm,
%            head=24true mm, top=24true mm, bottom=24true mm,
%           headheight=14true mm, headsep=10true mm
%           ]{geometry}% 东北林业大学要求的宽度

%\usepackage[a4paper,]{geometry}%四周都是24mm
\usepackage[a4paper,
left=24true mm,right=24true mm, top=25true mm, bottom=23true mm,
headheight=14true mm,
headsep=7true mm,
footskip=7true mm]{geometry}% 东北林业大学要求的宽度
\usepackage{layout}


\usepackage{titlesec}               % 控制标题的宏包
\usepackage{titletoc}                   % 控制目录的宏包
\usepackage{fancyhdr}                   % fancyhdr宏包 页眉和页脚的相关定义
%\usepackage[UTF8]{ctex}
\usepackage{color}          % 支持彩色
\usepackage{amsmath}        % AMSLaTeX宏包 用来排出更加漂亮的公式
\usepackage{amssymb}
\usepackage{epsfig}         % eps图像
\usepackage[below]{placeins}%允许上一个section 的浮动图形出现在下一个section的开始部分,还提供\FloatBarrier命令,使所有未处理的浮动图形立即被处理
\usepackage{flafter}       % 使得所有浮动体不能被放置在其浮动环境之前,以免浮动体在引述它的文本之前出现.
\usepackage{multirow}       %使用Multirow宏包,使得表格可以合并多个row 格
\usepackage{booktabs}       % 表格,横的粗线;\specialrule{1pt}{0pt}{0pt}
\usepackage{longtable}      %支持跨页的表格。
\usepackage{tabularx}
\usepackage{subfigure}%支持子图 %centerlast 设置最后一行是否居中
\usepackage[subfigure]{ccaption} %支持双语标题
\usepackage[sort&compress,numbers]{natbib}% 支持引用缩写的宏包
\makeatletter
%林大的参考文献研究用波浪链接~
\def\NAT@citexnum[#1][#2]#3{%
  \NAT@reset@parser
  \NAT@sort@cites{#3}%
  \NAT@reset@citea
  \@cite{\def\NAT@num{-1}\let\NAT@last@yr\relax\let\NAT@nm\@empty
    \@for\@citeb:=\NAT@cite@list\do
    {\@safe@activestrue
     \edef\@citeb{\expandafter\@firstofone\@citeb\@empty}%
     \@safe@activesfalse
     \@ifundefined{b@\@citeb\@extra@b@citeb}{%
       {\reset@font\bfseries?}
        \NAT@citeundefined\PackageWarning{natbib}%
       {Citation `\@citeb' on page \thepage \space undefined}}%
     {\let\NAT@last@num\NAT@num\let\NAT@last@nm\NAT@nm
      \NAT@parse{\@citeb}%
      \ifNAT@longnames\@ifundefined{bv@\@citeb\@extra@b@citeb}{%
        \let\NAT@name=\NAT@all@names
        \global\@namedef{bv@\@citeb\@extra@b@citeb}{}}{}%
      \fi
      \ifNAT@full\let\NAT@nm\NAT@all@names\else
        \let\NAT@nm\NAT@name\fi
      \ifNAT@swa
       \@ifnum{\NAT@ctype>\@ne}{%
        \@citea
        \NAT@hyper@{\@ifnum{\NAT@ctype=\tw@}{\NAT@test{\NAT@ctype}}{\NAT@alias}}%
       }{%
        \@ifnum{\NAT@cmprs>\z@}{%
         \NAT@ifcat@num\NAT@num
          {\let\NAT@nm=\NAT@num}%
          {\def\NAT@nm{-2}}%
         \NAT@ifcat@num\NAT@last@num
          {\@tempcnta=\NAT@last@num\relax}%
          {\@tempcnta\m@ne}%
         \@ifnum{\NAT@nm=\@tempcnta}{%
          \@ifnum{\NAT@merge>\@ne}{}{\NAT@last@yr@mbox}%
         }{%
           \advance\@tempcnta by\@ne
           \@ifnum{\NAT@nm=\@tempcnta}{%
             \ifx\NAT@last@yr\relax
               \def@NAT@last@yr{\@citea}%
             \else
               \def@NAT@last@yr{$\sim$\NAT@penalty}%
             \fi
           }{%
             \NAT@last@yr@mbox
           }%
         }%
        }{%
         \@tempswatrue
         \@ifnum{\NAT@merge>\@ne}{\@ifnum{\NAT@last@num=\NAT@num\relax}{\@tempswafalse}{}}{}%
         \if@tempswa\NAT@citea@mbox\fi
        }%
       }%
       \NAT@def@citea
      \else
        \ifcase\NAT@ctype
          \ifx\NAT@last@nm\NAT@nm \NAT@yrsep\NAT@penalty\NAT@space\else
            \@citea \NAT@test{\@ne}\NAT@spacechar\NAT@mbox{\NAT@super@kern\NAT@@open}%
          \fi
          \if*#1*\else#1\NAT@spacechar\fi
          \NAT@mbox{\NAT@hyper@{{\citenumfont{\NAT@num}}}}%
          \NAT@def@citea@box
        \or
          \NAT@hyper@citea@space{\NAT@test{\NAT@ctype}}%
        \or
          \NAT@hyper@citea@space{\NAT@test{\NAT@ctype}}%
        \or
          \NAT@hyper@citea@space\NAT@alias
        \fi
      \fi
     }%
    }%
      \@ifnum{\NAT@cmprs>\z@}{\NAT@last@yr}{}%
      \ifNAT@swa\else
        \@ifnum{\NAT@ctype=\z@}{%
          \if*#2*\else\NAT@cmt#2\fi
        }{}%
        \NAT@mbox{\NAT@@close}%
      \fi
  }{#1}{#2}%
}%
\makeatother

\usepackage{enumitem}       %使用enumitem宏包,改变列表项的格式
\usepackage{calc}           %长度可以用+ - * / 进行计算


%%\usepackage{txfonts}
\usepackage{mathtools}

\usepackage{bm}              % 处理数学公式中的黑斜体的宏包

\usepackage[xetex,
            bookmarksnumbered=true,
            bookmarksopen=true,
            colorlinks=false,
            pdfborder={0 0 1},
            citecolor=blue,
            linkcolor=red,
            anchorcolor=green,
            urlcolor=blue,
            breaklinks=true,
            naturalnames  %与algorithm2e宏包协调
            ]{hyperref}
%\usepackage{fontenc}
%xunicode xunicode Ross Moore’s xunicode package is now automatically loaded for users of both XELATEX and LuaLATEX.
\usepackage[BoldFont,SlantFont]{xeCJK}
\defaultfontfeatures{Mapping=tex-text}

\xeCJKsetemboldenfactor{1}%只对随后定义的CJK字体有效
\setCJKfamilyfont{hei}{SimHei}
\xeCJKsetemboldenfactor{4}
\setCJKfamilyfont{song}{SimSun}
\xeCJKsetemboldenfactor{1}
\setCJKfamilyfont{fs}{FangSong}
\setCJKfamilyfont{kai}{KaiTi}
\setCJKfamilyfont{li}{LiSu}
\setCJKfamilyfont{xw}{STXinwei}

\setCJKmainfont{SimSun}
\setmainfont{Times New Roman}
\setsansfont{Arial}

\newcommand{\hei}{\CJKfamily{hei}}% 黑体   (Windows自带simhei.ttf)
\newcommand{\song}{\CJKfamily{song}}    % 宋体   (Windows自带simsun.ttf)
\newcommand{\fs}{\CJKfamily{fs}}        % 仿宋体 (Windows自带simfs.ttf)
\newcommand{\kai}{\CJKfamily{kai}}      % 楷体   (Windows自带simkai.ttf)
\newcommand{\li}{\CJKfamily{li}}        % 隶书   (Windows自带simli.ttf)
\newcommand{\xw}{\CJKfamily{xw}}        % 隶书   (Windows自带simli.ttf)


\newfontfamily\arial{Arial}
\newfontfamily\timesnewroman{Times New Roman}

\usepackage[boxed,linesnumbered,algochapter]{algorithm2e}%\let\chapter\undefined
% 算法的宏包,注意宏包兼容性,先后顺序为float、hyperref、algorithm(2e),否则无法生成算法列表

\usepackage{setspace}%更改行距的宏包
%\usepackage{syntonly}检查语法不编译
%\syntaxonly
\usepackage{tikz}%R语言作图使用
\usepackage{xltxtra}%输出XeLaTeX
